% Options for packages loaded elsewhere
\PassOptionsToPackage{unicode}{hyperref}
\PassOptionsToPackage{hyphens}{url}
%
\documentclass[
]{article}
\usepackage{amsmath,amssymb}
\usepackage{iftex}
\ifPDFTeX
  \usepackage[T1]{fontenc}
  \usepackage[utf8]{inputenc}
  \usepackage{textcomp} % provide euro and other symbols
\else % if luatex or xetex
  \usepackage{unicode-math} % this also loads fontspec
  \defaultfontfeatures{Scale=MatchLowercase}
  \defaultfontfeatures[\rmfamily]{Ligatures=TeX,Scale=1}
\fi
\usepackage{lmodern}
\ifPDFTeX\else
  % xetex/luatex font selection
\fi
% Use upquote if available, for straight quotes in verbatim environments
\IfFileExists{upquote.sty}{\usepackage{upquote}}{}
\IfFileExists{microtype.sty}{% use microtype if available
  \usepackage[]{microtype}
  \UseMicrotypeSet[protrusion]{basicmath} % disable protrusion for tt fonts
}{}
\makeatletter
\@ifundefined{KOMAClassName}{% if non-KOMA class
  \IfFileExists{parskip.sty}{%
    \usepackage{parskip}
  }{% else
    \setlength{\parindent}{0pt}
    \setlength{\parskip}{6pt plus 2pt minus 1pt}}
}{% if KOMA class
  \KOMAoptions{parskip=half}}
\makeatother
\usepackage{xcolor}
\usepackage[margin=1in]{geometry}
\usepackage{color}
\usepackage{fancyvrb}
\newcommand{\VerbBar}{|}
\newcommand{\VERB}{\Verb[commandchars=\\\{\}]}
\DefineVerbatimEnvironment{Highlighting}{Verbatim}{commandchars=\\\{\}}
% Add ',fontsize=\small' for more characters per line
\usepackage{framed}
\definecolor{shadecolor}{RGB}{248,248,248}
\newenvironment{Shaded}{\begin{snugshade}}{\end{snugshade}}
\newcommand{\AlertTok}[1]{\textcolor[rgb]{0.94,0.16,0.16}{#1}}
\newcommand{\AnnotationTok}[1]{\textcolor[rgb]{0.56,0.35,0.01}{\textbf{\textit{#1}}}}
\newcommand{\AttributeTok}[1]{\textcolor[rgb]{0.13,0.29,0.53}{#1}}
\newcommand{\BaseNTok}[1]{\textcolor[rgb]{0.00,0.00,0.81}{#1}}
\newcommand{\BuiltInTok}[1]{#1}
\newcommand{\CharTok}[1]{\textcolor[rgb]{0.31,0.60,0.02}{#1}}
\newcommand{\CommentTok}[1]{\textcolor[rgb]{0.56,0.35,0.01}{\textit{#1}}}
\newcommand{\CommentVarTok}[1]{\textcolor[rgb]{0.56,0.35,0.01}{\textbf{\textit{#1}}}}
\newcommand{\ConstantTok}[1]{\textcolor[rgb]{0.56,0.35,0.01}{#1}}
\newcommand{\ControlFlowTok}[1]{\textcolor[rgb]{0.13,0.29,0.53}{\textbf{#1}}}
\newcommand{\DataTypeTok}[1]{\textcolor[rgb]{0.13,0.29,0.53}{#1}}
\newcommand{\DecValTok}[1]{\textcolor[rgb]{0.00,0.00,0.81}{#1}}
\newcommand{\DocumentationTok}[1]{\textcolor[rgb]{0.56,0.35,0.01}{\textbf{\textit{#1}}}}
\newcommand{\ErrorTok}[1]{\textcolor[rgb]{0.64,0.00,0.00}{\textbf{#1}}}
\newcommand{\ExtensionTok}[1]{#1}
\newcommand{\FloatTok}[1]{\textcolor[rgb]{0.00,0.00,0.81}{#1}}
\newcommand{\FunctionTok}[1]{\textcolor[rgb]{0.13,0.29,0.53}{\textbf{#1}}}
\newcommand{\ImportTok}[1]{#1}
\newcommand{\InformationTok}[1]{\textcolor[rgb]{0.56,0.35,0.01}{\textbf{\textit{#1}}}}
\newcommand{\KeywordTok}[1]{\textcolor[rgb]{0.13,0.29,0.53}{\textbf{#1}}}
\newcommand{\NormalTok}[1]{#1}
\newcommand{\OperatorTok}[1]{\textcolor[rgb]{0.81,0.36,0.00}{\textbf{#1}}}
\newcommand{\OtherTok}[1]{\textcolor[rgb]{0.56,0.35,0.01}{#1}}
\newcommand{\PreprocessorTok}[1]{\textcolor[rgb]{0.56,0.35,0.01}{\textit{#1}}}
\newcommand{\RegionMarkerTok}[1]{#1}
\newcommand{\SpecialCharTok}[1]{\textcolor[rgb]{0.81,0.36,0.00}{\textbf{#1}}}
\newcommand{\SpecialStringTok}[1]{\textcolor[rgb]{0.31,0.60,0.02}{#1}}
\newcommand{\StringTok}[1]{\textcolor[rgb]{0.31,0.60,0.02}{#1}}
\newcommand{\VariableTok}[1]{\textcolor[rgb]{0.00,0.00,0.00}{#1}}
\newcommand{\VerbatimStringTok}[1]{\textcolor[rgb]{0.31,0.60,0.02}{#1}}
\newcommand{\WarningTok}[1]{\textcolor[rgb]{0.56,0.35,0.01}{\textbf{\textit{#1}}}}
\usepackage{graphicx}
\makeatletter
\def\maxwidth{\ifdim\Gin@nat@width>\linewidth\linewidth\else\Gin@nat@width\fi}
\def\maxheight{\ifdim\Gin@nat@height>\textheight\textheight\else\Gin@nat@height\fi}
\makeatother
% Scale images if necessary, so that they will not overflow the page
% margins by default, and it is still possible to overwrite the defaults
% using explicit options in \includegraphics[width, height, ...]{}
\setkeys{Gin}{width=\maxwidth,height=\maxheight,keepaspectratio}
% Set default figure placement to htbp
\makeatletter
\def\fps@figure{htbp}
\makeatother
\setlength{\emergencystretch}{3em} % prevent overfull lines
\providecommand{\tightlist}{%
  \setlength{\itemsep}{0pt}\setlength{\parskip}{0pt}}
\setcounter{secnumdepth}{-\maxdimen} % remove section numbering
\ifLuaTeX
  \usepackage{selnolig}  % disable illegal ligatures
\fi
\usepackage{bookmark}
\IfFileExists{xurl.sty}{\usepackage{xurl}}{} % add URL line breaks if available
\urlstyle{same}
\hypersetup{
  hidelinks,
  pdfcreator={LaTeX via pandoc}}

\author{}
\date{\vspace{-2.5em}}

\begin{document}

There are 3-4 packages you will need to install for today's practical:
\texttt{install.packages(c("xgboost",\ "eegkit",\ "forecast",\ "tseries",\ "caret"))}
apart from that everything else should already be available on your
system.

If you are using a newer Mac you may have to also install
\href{https://www.xquartz.org/}{quartz} to have everything work (do this
if you see errors about \texttt{X11} during install/execution).

I will endeavour to use explicit imports to make it clear where
functions are coming from (functions without \texttt{library\_name::}
are part of base R or a function we've defined in this notebook).

\begin{verbatim}
## Warning: package 'xgboost' was built under R version 4.3.3
\end{verbatim}

\begin{verbatim}
## Warning: package 'eegkit' was built under R version 4.3.3
\end{verbatim}

\begin{verbatim}
## Loading required package: eegkitdata
\end{verbatim}

\begin{verbatim}
## Warning: package 'eegkitdata' was built under R version 4.3.1
\end{verbatim}

\begin{verbatim}
## Loading required package: bigsplines
\end{verbatim}

\begin{verbatim}
## Warning: package 'bigsplines' was built under R version 4.3.1
\end{verbatim}

\begin{verbatim}
## Loading required package: quadprog
\end{verbatim}

\begin{verbatim}
## Warning: package 'quadprog' was built under R version 4.3.1
\end{verbatim}

\begin{verbatim}
## Loading required package: ica
\end{verbatim}

\begin{verbatim}
## Warning: package 'ica' was built under R version 4.3.1
\end{verbatim}

\begin{verbatim}
## Loading required package: rgl
\end{verbatim}

\begin{verbatim}
## Warning: package 'rgl' was built under R version 4.3.3
\end{verbatim}

\begin{verbatim}
## Loading required package: signal
\end{verbatim}

\begin{verbatim}
## Warning: package 'signal' was built under R version 4.3.3
\end{verbatim}

\begin{verbatim}
## 
## Attaching package: 'signal'
\end{verbatim}

\begin{verbatim}
## The following objects are masked from 'package:stats':
## 
##     filter, poly
\end{verbatim}

\begin{verbatim}
## Warning: package 'forecast' was built under R version 4.3.3
\end{verbatim}

\begin{verbatim}
## Registered S3 method overwritten by 'quantmod':
##   method            from
##   as.zoo.data.frame zoo
\end{verbatim}

\begin{verbatim}
## Warning: package 'tseries' was built under R version 4.3.3
\end{verbatim}

\begin{verbatim}
## Warning: package 'caret' was built under R version 4.3.1
\end{verbatim}

\begin{verbatim}
## Loading required package: ggplot2
\end{verbatim}

\begin{verbatim}
## Warning: package 'ggplot2' was built under R version 4.3.3
\end{verbatim}

\begin{verbatim}
## Loading required package: lattice
\end{verbatim}

\begin{verbatim}
## Warning: package 'lattice' was built under R version 4.3.3
\end{verbatim}

\begin{verbatim}
## Warning: package 'dplyr' was built under R version 4.3.3
\end{verbatim}

\begin{verbatim}
## 
## Attaching package: 'dplyr'
\end{verbatim}

\begin{verbatim}
## The following object is masked from 'package:signal':
## 
##     filter
\end{verbatim}

\begin{verbatim}
## The following object is masked from 'package:xgboost':
## 
##     slice
\end{verbatim}

\begin{verbatim}
## The following objects are masked from 'package:stats':
## 
##     filter, lag
\end{verbatim}

\begin{verbatim}
## The following objects are masked from 'package:base':
## 
##     intersect, setdiff, setequal, union
\end{verbatim}

\begin{verbatim}
## Warning: package 'purrr' was built under R version 4.3.3
\end{verbatim}

\begin{verbatim}
## 
## Attaching package: 'purrr'
\end{verbatim}

\begin{verbatim}
## The following object is masked from 'package:caret':
## 
##     lift
\end{verbatim}

\subsection{EEG Eye Detection Data}\label{eeg-eye-detection-data}

One of the most common types of medical sensor data (and one that we
talked about during the lecture) are Electroencephalograms (EEGs).\\
These measure mesoscale electrical signals (measured in microvolts)
within the brain, which are indicative of a region of neuronal activity.
Typically, EEGs involve an array of sensors (aka channels) placed on the
scalp with a high degree of covariance between sensors.

As EEG data can be very large and unwieldy, we are going to use a
relatively small/simple dataset today from
\href{http://ehrai.com/su/pdf/aihls2013.pdf}{this paper}.

This dataset is a 117 second continuous EEG measurement collected from a
single person with a device called a ``Emotiv EEG Neuroheadset''. In
combination with the EEG data collection, a camera was used to record
whether person being recorded had their eyes open or closed. This was
eye status was then manually annotated onto the EEG data with \texttt{1}
indicated the eyes being closed and \texttt{0} the eyes being open.
Measures microvoltages are listed in chronological order with the first
measured value at the top of the dataframe.

Let's parse the data directly from the \texttt{h2o} library's (which we
aren't actually using directly) test data S3 bucket:

\begin{Shaded}
\begin{Highlighting}[]
\NormalTok{eeg\_url }\OtherTok{\textless{}{-}} \StringTok{"https://h2o{-}public{-}test{-}data.s3.amazonaws.com/smalldata/eeg/eeg\_eyestate\_splits.csv"}
\NormalTok{eeg\_data }\OtherTok{\textless{}{-}} \FunctionTok{read.csv}\NormalTok{(eeg\_url)}

\CommentTok{\# add timestamp}
\NormalTok{Fs }\OtherTok{\textless{}{-}} \DecValTok{117} \SpecialCharTok{/} \FunctionTok{nrow}\NormalTok{(eeg\_data)}
\NormalTok{eeg\_data }\OtherTok{\textless{}{-}} \FunctionTok{transform}\NormalTok{(eeg\_data, }\AttributeTok{ds =} \FunctionTok{seq}\NormalTok{(}\DecValTok{0}\NormalTok{, }\FloatTok{116.99999}\NormalTok{, }\AttributeTok{by =}\NormalTok{ Fs), }\AttributeTok{eyeDetection =} \FunctionTok{as.factor}\NormalTok{(eyeDetection))}
\FunctionTok{print}\NormalTok{(}\FunctionTok{table}\NormalTok{(eeg\_data}\SpecialCharTok{$}\NormalTok{eyeDetection))}
\end{Highlighting}
\end{Shaded}

\begin{verbatim}
## 
##    0    1 
## 8257 6723
\end{verbatim}

\begin{Shaded}
\begin{Highlighting}[]
\CommentTok{\# split dataset into train, validate, test}
\NormalTok{eeg\_train }\OtherTok{\textless{}{-}} \FunctionTok{subset}\NormalTok{(eeg\_data, split }\SpecialCharTok{==} \StringTok{\textquotesingle{}train\textquotesingle{}}\NormalTok{, }\AttributeTok{select =} \SpecialCharTok{{-}}\NormalTok{split)}
\FunctionTok{print}\NormalTok{(}\FunctionTok{table}\NormalTok{(eeg\_train}\SpecialCharTok{$}\NormalTok{eyeDetection))}
\end{Highlighting}
\end{Shaded}

\begin{verbatim}
## 
##    0    1 
## 4916 4072
\end{verbatim}

\begin{Shaded}
\begin{Highlighting}[]
\NormalTok{eeg\_validate }\OtherTok{\textless{}{-}} \FunctionTok{subset}\NormalTok{(eeg\_data, split }\SpecialCharTok{==} \StringTok{\textquotesingle{}valid\textquotesingle{}}\NormalTok{, }\AttributeTok{select =} \SpecialCharTok{{-}}\NormalTok{split)}
\NormalTok{eeg\_test }\OtherTok{\textless{}{-}} \FunctionTok{subset}\NormalTok{(eeg\_data, split }\SpecialCharTok{==} \StringTok{\textquotesingle{}test\textquotesingle{}}\NormalTok{, }\AttributeTok{select =} \SpecialCharTok{{-}}\NormalTok{split)}
\end{Highlighting}
\end{Shaded}

\textbf{0} Knowing the \texttt{eeg\_data} contains 117 seconds of data,
inspect the \texttt{eeg\_data} dataframe and the code above to and
determine how many samples per second were taken?

Samples per second = 14980/117 = 128

\textbf{1} How many EEG electrodes/sensors were used?

\begin{Shaded}
\begin{Highlighting}[]
\FunctionTok{colnames}\NormalTok{(eeg\_data)}
\end{Highlighting}
\end{Shaded}

\begin{verbatim}
##  [1] "AF3"          "F7"           "F3"           "FC5"          "T7"          
##  [6] "P7"           "O1"           "O2"           "P8"           "T8"          
## [11] "FC6"          "F4"           "F8"           "AF4"          "eyeDetection"
## [16] "split"        "ds"
\end{verbatim}

\begin{Shaded}
\begin{Highlighting}[]
\NormalTok{sensor\_columns }\OtherTok{\textless{}{-}} \FunctionTok{setdiff}\NormalTok{(}\FunctionTok{colnames}\NormalTok{(eeg\_data), }\FunctionTok{c}\NormalTok{(}\StringTok{"split"}\NormalTok{, }\StringTok{"ds"}\NormalTok{, }\StringTok{"eyeDetection"}\NormalTok{))}
\NormalTok{num\_sensors }\OtherTok{\textless{}{-}} \FunctionTok{length}\NormalTok{(sensor\_columns)}
\end{Highlighting}
\end{Shaded}

\subsubsection{Exploratory Data
Analysis}\label{exploratory-data-analysis}

Now that we have the dataset and some basic parameters let's begin with
the ever important/relevant exploratory data analysis.

First we should check there is no missing data!

\begin{Shaded}
\begin{Highlighting}[]
\FunctionTok{sum}\NormalTok{(}\FunctionTok{is.na}\NormalTok{(eeg\_data))}
\end{Highlighting}
\end{Shaded}

\begin{verbatim}
## [1] 0
\end{verbatim}

Great, now we can start generating some plots to look at this data
within the time-domain.

First we use \texttt{reshape2::melt()} to transform the
\texttt{eeg\_data} dataset from a wide format to a long format expected
by \texttt{ggplot2}.

Specifically, this converts from ``wide'' where each electrode has its
own column, to a ``long'' format, where each observation has its own
row. This format is often more convenient for data analysis and
visualization, especially when dealing with repeated measurements or
time-series data.

We then use \texttt{ggplot2} to create a line plot of electrode
intensities per sampling time, with the lines coloured by electrode, and
the eye status annotated using dark grey blocks.

\begin{Shaded}
\begin{Highlighting}[]
\NormalTok{melt }\OtherTok{\textless{}{-}}\NormalTok{ reshape2}\SpecialCharTok{::}\FunctionTok{melt}\NormalTok{(eeg\_data }\SpecialCharTok{\%\textgreater{}\%}\NormalTok{ dplyr}\SpecialCharTok{::}\FunctionTok{select}\NormalTok{(}\SpecialCharTok{{-}}\NormalTok{split), }\AttributeTok{id.vars=}\FunctionTok{c}\NormalTok{(}\StringTok{"eyeDetection"}\NormalTok{, }\StringTok{"ds"}\NormalTok{), }\AttributeTok{variable.name =} \StringTok{"Electrode"}\NormalTok{, }\AttributeTok{value.name =} \StringTok{"microvolts"}\NormalTok{)}


\NormalTok{ggplot2}\SpecialCharTok{::}\FunctionTok{ggplot}\NormalTok{(melt, ggplot2}\SpecialCharTok{::}\FunctionTok{aes}\NormalTok{(}\AttributeTok{x=}\NormalTok{ds, }\AttributeTok{y=}\NormalTok{microvolts, }\AttributeTok{color=}\NormalTok{Electrode)) }\SpecialCharTok{+} 
\NormalTok{  ggplot2}\SpecialCharTok{::}\FunctionTok{geom\_line}\NormalTok{() }\SpecialCharTok{+} 
\NormalTok{  ggplot2}\SpecialCharTok{::}\FunctionTok{ylim}\NormalTok{(}\DecValTok{3500}\NormalTok{,}\DecValTok{5000}\NormalTok{) }\SpecialCharTok{+} 
\NormalTok{  ggplot2}\SpecialCharTok{::}\FunctionTok{geom\_vline}\NormalTok{(ggplot2}\SpecialCharTok{::}\FunctionTok{aes}\NormalTok{(}\AttributeTok{xintercept=}\NormalTok{ds), }\AttributeTok{data=}\NormalTok{dplyr}\SpecialCharTok{::}\FunctionTok{filter}\NormalTok{(melt, eyeDetection}\SpecialCharTok{==}\DecValTok{1}\NormalTok{), }\AttributeTok{alpha=}\FloatTok{0.005}\NormalTok{)}
\end{Highlighting}
\end{Shaded}

\includegraphics{lab4_sensor_XGboost_files/figure-latex/plot_data-1.pdf}

\textbf{2} Do you see any obvious patterns between eyes being open (dark
grey blocks in the plot) and the EEG intensities?

Amplitude Differences: There appear to be variations in the amplitude of
the EEG signals when the eyes are closed compared to when they are open.
The signals generally show increased amplitude during the periods of eye
closure (dark grey blocks). Consistent Patterns Across Electrodes: Many
electrodes has consistent changes in the EEG signal when the eyes are
closed. Distinct Episodes: The dark grey blocks (indicating eyes closed)
correspond to more pronounced and synchronized changes in EEG There are
shifts in the baseline of some channels, which might be indicative of
different levels of neuronal activity or noise artifacts.

\textbf{3} Similarly, based on the distribution of eye open/close state
over time to anticipate any temporal correlation between these states?

\begin{Shaded}
\begin{Highlighting}[]
\NormalTok{ggplot2}\SpecialCharTok{::}\FunctionTok{ggplot}\NormalTok{(eeg\_data, ggplot2}\SpecialCharTok{::}\FunctionTok{aes}\NormalTok{(}\AttributeTok{x =}\NormalTok{ ds, }\AttributeTok{y =} \FunctionTok{as.numeric}\NormalTok{(eyeDetection))) }\SpecialCharTok{+} 
\NormalTok{  ggplot2}\SpecialCharTok{::}\FunctionTok{geom\_line}\NormalTok{() }\SpecialCharTok{+} 
\NormalTok{  ggplot2}\SpecialCharTok{::}\FunctionTok{labs}\NormalTok{(}\AttributeTok{x =} \StringTok{"Time (seconds)"}\NormalTok{, }\AttributeTok{y =} \StringTok{"Eye State (0=open, 1=closed)"}\NormalTok{) }\SpecialCharTok{+} 
\NormalTok{  ggplot2}\SpecialCharTok{::}\FunctionTok{ggtitle}\NormalTok{(}\StringTok{"Eye State Over Time"}\NormalTok{)}
\end{Highlighting}
\end{Shaded}

\includegraphics{lab4_sensor_XGboost_files/figure-latex/unnamed-chunk-2-1.pdf}

\begin{Shaded}
\begin{Highlighting}[]
\NormalTok{eye\_state\_numeric }\OtherTok{\textless{}{-}} \FunctionTok{as.numeric}\NormalTok{(eeg\_data}\SpecialCharTok{$}\NormalTok{eyeDetection) }\SpecialCharTok{{-}} \DecValTok{1}  \CommentTok{\# convert factor to numeric}
\NormalTok{acf\_result }\OtherTok{\textless{}{-}} \FunctionTok{acf}\NormalTok{(eye\_state\_numeric, }\AttributeTok{lag.max =} \DecValTok{100}\NormalTok{, }\AttributeTok{plot =} \ConstantTok{TRUE}\NormalTok{)}
\end{Highlighting}
\end{Shaded}

\includegraphics{lab4_sensor_XGboost_files/figure-latex/unnamed-chunk-2-2.pdf}
ACF Plot: Peaks at regular intervals in the ACF plot suggest
periodicity. If the ACF decays slowly, it indicates that past values
have a lasting influence on future values (positive correlation). If it
oscillates, there might be a periodic pattern.

Let's see if we can directly look at the distribution of EEG intensities
and see how they related to eye status.

As there are a few extreme outliers in voltage we will use the
\texttt{dplyr::filter} function to remove values outwith of 3750 to
50003. The function uses the \texttt{\%in\%} operator to check if each
value of microvolts is within that range. The function also uses the
\texttt{dplyr::mutate()} to change the type of the variable eyeDetection
from numeric to a factor (R's categorical variable type).

\begin{Shaded}
\begin{Highlighting}[]
\NormalTok{melt\_train }\OtherTok{\textless{}{-}}\NormalTok{ reshape2}\SpecialCharTok{::}\FunctionTok{melt}\NormalTok{(eeg\_train, }\AttributeTok{id.vars=}\FunctionTok{c}\NormalTok{(}\StringTok{"eyeDetection"}\NormalTok{, }\StringTok{"ds"}\NormalTok{), }\AttributeTok{variable.name =} \StringTok{"Electrode"}\NormalTok{, }\AttributeTok{value.name =} \StringTok{"microvolts"}\NormalTok{)}

\CommentTok{\# filter huge outliers in voltage}
\NormalTok{filt\_melt\_train }\OtherTok{\textless{}{-}}\NormalTok{ dplyr}\SpecialCharTok{::}\FunctionTok{filter}\NormalTok{(melt\_train, microvolts }\SpecialCharTok{\%in\%}\NormalTok{ (}\DecValTok{3750}\SpecialCharTok{:}\DecValTok{5000}\NormalTok{)) }\SpecialCharTok{\%\textgreater{}\%}\NormalTok{ dplyr}\SpecialCharTok{::}\FunctionTok{mutate}\NormalTok{(}\AttributeTok{eyeDetection=}\FunctionTok{as.factor}\NormalTok{(eyeDetection))}

\NormalTok{ggplot2}\SpecialCharTok{::}\FunctionTok{ggplot}\NormalTok{(filt\_melt\_train, ggplot2}\SpecialCharTok{::}\FunctionTok{aes}\NormalTok{(}\AttributeTok{y=}\NormalTok{Electrode, }\AttributeTok{x=}\NormalTok{microvolts, }\AttributeTok{fill=}\NormalTok{eyeDetection)) }\SpecialCharTok{+}\NormalTok{ ggplot2}\SpecialCharTok{::}\FunctionTok{geom\_boxplot}\NormalTok{()}
\end{Highlighting}
\end{Shaded}

\includegraphics{lab4_sensor_XGboost_files/figure-latex/compare_distrib-1.pdf}

Plots are great but sometimes so it is also useful to directly look at
the summary statistics and how they related to eye status. We will do
this by grouping the data based on eye status and electrode before
calculating the statistics using the convenient
\texttt{dplyr::summarise} function.

\begin{Shaded}
\begin{Highlighting}[]
\NormalTok{filt\_melt\_train }\SpecialCharTok{\%\textgreater{}\%}\NormalTok{ dplyr}\SpecialCharTok{::}\FunctionTok{group\_by}\NormalTok{(eyeDetection, Electrode) }\SpecialCharTok{\%\textgreater{}\%} 
\NormalTok{    dplyr}\SpecialCharTok{::}\FunctionTok{summarise}\NormalTok{(}\AttributeTok{mean =} \FunctionTok{mean}\NormalTok{(microvolts), }\AttributeTok{median=}\FunctionTok{median}\NormalTok{(microvolts), }\AttributeTok{sd=}\FunctionTok{sd}\NormalTok{(microvolts)) }\SpecialCharTok{\%\textgreater{}\%} 
\NormalTok{    dplyr}\SpecialCharTok{::}\FunctionTok{arrange}\NormalTok{(Electrode)}
\end{Highlighting}
\end{Shaded}

\begin{verbatim}
## `summarise()` has grouped output by 'eyeDetection'. You can override using the
## `.groups` argument.
\end{verbatim}

\begin{verbatim}
## # A tibble: 28 x 5
## # Groups:   eyeDetection [2]
##    eyeDetection Electrode  mean median    sd
##    <fct>        <fct>     <dbl>  <dbl> <dbl>
##  1 0            AF3       4294.   4300  35.4
##  2 1            AF3       4305.   4300  34.4
##  3 0            F7        4015.   4020  28.4
##  4 1            F7        4007.   4000  24.9
##  5 0            F3        4268.   4260  20.9
##  6 1            F3        4269.   4260  17.4
##  7 0            FC5       4124.   4120  17.3
##  8 1            FC5       4124.   4120  19.2
##  9 0            T7        4341.   4340  13.9
## 10 1            T7        4342.   4340  15.5
## # i 18 more rows
\end{verbatim}

\textbf{4} Based on these analyses are any electrodes consistently more
intense or varied when eyes are open? electrodes consistnently showing
either higher intensity or variability (standard deviation) when eyes
are open are: AF3 F3 T7 O2 P8 FC6 F4 F8 \#\#\#\# Time-Related Trends

As it looks like there may be a temporal pattern in the data we should
investigate how it changes over time.

First we will do a statistical test for stationarity:

\begin{Shaded}
\begin{Highlighting}[]
\FunctionTok{apply}\NormalTok{(eeg\_train, }\DecValTok{2}\NormalTok{, tseries}\SpecialCharTok{::}\NormalTok{adf.test)}
\end{Highlighting}
\end{Shaded}

\begin{verbatim}
## Warning in FUN(newX[, i], ...): p-value smaller than printed p-value
## Warning in FUN(newX[, i], ...): p-value smaller than printed p-value
## Warning in FUN(newX[, i], ...): p-value smaller than printed p-value
## Warning in FUN(newX[, i], ...): p-value smaller than printed p-value
## Warning in FUN(newX[, i], ...): p-value smaller than printed p-value
## Warning in FUN(newX[, i], ...): p-value smaller than printed p-value
## Warning in FUN(newX[, i], ...): p-value smaller than printed p-value
## Warning in FUN(newX[, i], ...): p-value smaller than printed p-value
## Warning in FUN(newX[, i], ...): p-value smaller than printed p-value
## Warning in FUN(newX[, i], ...): p-value smaller than printed p-value
## Warning in FUN(newX[, i], ...): p-value smaller than printed p-value
## Warning in FUN(newX[, i], ...): p-value smaller than printed p-value
## Warning in FUN(newX[, i], ...): p-value smaller than printed p-value
## Warning in FUN(newX[, i], ...): p-value smaller than printed p-value
## Warning in FUN(newX[, i], ...): p-value smaller than printed p-value
\end{verbatim}

\begin{verbatim}
## $AF3
## 
##  Augmented Dickey-Fuller Test
## 
## data:  newX[, i]
## Dickey-Fuller = -20.669, Lag order = 20, p-value = 0.01
## alternative hypothesis: stationary
## 
## 
## $F7
## 
##  Augmented Dickey-Fuller Test
## 
## data:  newX[, i]
## Dickey-Fuller = -12.079, Lag order = 20, p-value = 0.01
## alternative hypothesis: stationary
## 
## 
## $F3
## 
##  Augmented Dickey-Fuller Test
## 
## data:  newX[, i]
## Dickey-Fuller = -11.587, Lag order = 20, p-value = 0.01
## alternative hypothesis: stationary
## 
## 
## $FC5
## 
##  Augmented Dickey-Fuller Test
## 
## data:  newX[, i]
## Dickey-Fuller = -11.122, Lag order = 20, p-value = 0.01
## alternative hypothesis: stationary
## 
## 
## $T7
## 
##  Augmented Dickey-Fuller Test
## 
## data:  newX[, i]
## Dickey-Fuller = -9.5644, Lag order = 20, p-value = 0.01
## alternative hypothesis: stationary
## 
## 
## $P7
## 
##  Augmented Dickey-Fuller Test
## 
## data:  newX[, i]
## Dickey-Fuller = -20.7, Lag order = 20, p-value = 0.01
## alternative hypothesis: stationary
## 
## 
## $O1
## 
##  Augmented Dickey-Fuller Test
## 
## data:  newX[, i]
## Dickey-Fuller = -7.9495, Lag order = 20, p-value = 0.01
## alternative hypothesis: stationary
## 
## 
## $O2
## 
##  Augmented Dickey-Fuller Test
## 
## data:  newX[, i]
## Dickey-Fuller = -9.3537, Lag order = 20, p-value = 0.01
## alternative hypothesis: stationary
## 
## 
## $P8
## 
##  Augmented Dickey-Fuller Test
## 
## data:  newX[, i]
## Dickey-Fuller = -20.69, Lag order = 20, p-value = 0.01
## alternative hypothesis: stationary
## 
## 
## $T8
## 
##  Augmented Dickey-Fuller Test
## 
## data:  newX[, i]
## Dickey-Fuller = -9.9902, Lag order = 20, p-value = 0.01
## alternative hypothesis: stationary
## 
## 
## $FC6
## 
##  Augmented Dickey-Fuller Test
## 
## data:  newX[, i]
## Dickey-Fuller = -8.6708, Lag order = 20, p-value = 0.01
## alternative hypothesis: stationary
## 
## 
## $F4
## 
##  Augmented Dickey-Fuller Test
## 
## data:  newX[, i]
## Dickey-Fuller = -10.189, Lag order = 20, p-value = 0.01
## alternative hypothesis: stationary
## 
## 
## $F8
## 
##  Augmented Dickey-Fuller Test
## 
## data:  newX[, i]
## Dickey-Fuller = -20.642, Lag order = 20, p-value = 0.01
## alternative hypothesis: stationary
## 
## 
## $AF4
## 
##  Augmented Dickey-Fuller Test
## 
## data:  newX[, i]
## Dickey-Fuller = -20.755, Lag order = 20, p-value = 0.01
## alternative hypothesis: stationary
## 
## 
## $eyeDetection
## 
##  Augmented Dickey-Fuller Test
## 
## data:  newX[, i]
## Dickey-Fuller = -4.8699, Lag order = 20, p-value = 0.01
## alternative hypothesis: stationary
## 
## 
## $ds
## 
##  Augmented Dickey-Fuller Test
## 
## data:  newX[, i]
## Dickey-Fuller = -2.4104, Lag order = 20, p-value = 0.4045
## alternative hypothesis: stationary
\end{verbatim}

\textbf{5} What is stationarity? ADF test is used to test for
stationarity in each variable (electrodes, eye detection, and ``ds''
variable) of the EEG data. The null hypothesis of the ADF test is that
the time series has a unit root, indicating non-stationarity, while the
alternative hypothesis is that the series is stationary. A low p-value
(typically below a significance level, often 0.05) suggests rejecting
the null hypothesis in favor of stationarity.

Based on the ADF test results, all variables except for the ``ds''
variable show evidence of stationarity, as indicated by the low
p-values. Therefore, the EEG data for the electrodes and eye detection
appear to exhibit stationary behavior, which means their statistical
properties do not significantly change over time.

\textbf{6} Why are we interested in stationarity? What do the results of
these tests tell us? (ignoring the lack of multiple comparison
correction\ldots)

Stationary time series allow for more straightforward interpretation of
statistical measures such as mean, variance, and autocorrelation.
Changes in these measures over time can indicate real shifts in the
underlying process rather than artifacts of non-stationarity. The
variables (electrodes and eye detection) showing evidence of
stationarity suggest that their statistical properties remain consistent
over time. This stability allows for reliable analysis and
interpretation of these variables without concerns about evolving
patterns or trends.

The results of the ADF tests tell us that most of the EEG channels are
stationary (p-value = 0.01), meaning their properties are stable over
time. However, the time variable (`ds') is non-stationary (p-value =
0.4045), which is expected as time itself progresses and is not
constant.

Then we may want to visually explore patterns of autocorrelation
(previous values predict future ones) and cross-correlation (correlation
across channels over time) using \texttt{forecast::ggAcf} function.

The ACF plot displays the cross- and auto-correlation values for
different lags (i.e., time delayed versions of each electrode's voltage
timeseries) in the dataset. It helps identify any significant
correlations between channels and observations at different time points.
Positive autocorrelation indicates that the increase in voltage observed
in a given time-interval leads to a proportionate increase in the lagged
time interval as well. Negative autocorrelation indicates the opposite!

\begin{Shaded}
\begin{Highlighting}[]
\NormalTok{forecast}\SpecialCharTok{::}\FunctionTok{ggAcf}\NormalTok{(eeg\_train }\SpecialCharTok{\%\textgreater{}\%}\NormalTok{ dplyr}\SpecialCharTok{::}\FunctionTok{select}\NormalTok{(}\SpecialCharTok{{-}}\NormalTok{ds))}
\end{Highlighting}
\end{Shaded}

\includegraphics{lab4_sensor_XGboost_files/figure-latex/correlation-1.pdf}

\textbf{7} Do any fields show signs of strong autocorrelation (diagonal
plots)? Do any pairs of fields show signs of cross-correlation? Provide
examples. Channels such as F7, F3, FC5, and several others show strong
signs of positive autocorrelation. eyeDetection shows strong
autocorrelation, suggesting that the state of the eyes (open or closed)
has a consistent pattern over time. F3 and F4, FC5 and FC6, and other
adjacent or similarly positioned electrodes might show cross-correlation
due to their physical proximity on the scalp and the similar brain
activity they measure.

\paragraph{Frequency-Space}\label{frequency-space}

We can also explore the data in frequency space by using a Fast Fourier
Transform.\\
After the FFT we can summarise the distributions of frequencies by their
density across the power spectrum. This will let us see if there any
obvious patterns related to eye status in the overall frequency
distributions.

\begin{Shaded}
\begin{Highlighting}[]
\NormalTok{eegkit}\SpecialCharTok{::}\FunctionTok{eegpsd}\NormalTok{(eeg\_train }\SpecialCharTok{\%\textgreater{}\%}\NormalTok{ dplyr}\SpecialCharTok{::}\FunctionTok{filter}\NormalTok{(eyeDetection }\SpecialCharTok{==} \DecValTok{0}\NormalTok{) }\SpecialCharTok{\%\textgreater{}\%}\NormalTok{ dplyr}\SpecialCharTok{::}\FunctionTok{select}\NormalTok{(}\SpecialCharTok{{-}}\NormalTok{eyeDetection, }\SpecialCharTok{{-}}\NormalTok{ds), }\AttributeTok{Fs =}\NormalTok{ Fs, }\AttributeTok{xlab=}\StringTok{"Eye Open"}\NormalTok{)}
\end{Highlighting}
\end{Shaded}

\includegraphics{lab4_sensor_XGboost_files/figure-latex/fft_open-1.pdf}

\begin{Shaded}
\begin{Highlighting}[]
\NormalTok{eegkit}\SpecialCharTok{::}\FunctionTok{eegpsd}\NormalTok{(eeg\_train }\SpecialCharTok{\%\textgreater{}\%}\NormalTok{ dplyr}\SpecialCharTok{::}\FunctionTok{filter}\NormalTok{(eyeDetection }\SpecialCharTok{==} \DecValTok{1}\NormalTok{) }\SpecialCharTok{\%\textgreater{}\%}\NormalTok{ dplyr}\SpecialCharTok{::}\FunctionTok{select}\NormalTok{(}\SpecialCharTok{{-}}\NormalTok{eyeDetection, }\SpecialCharTok{{-}}\NormalTok{ds), }\AttributeTok{Fs =}\NormalTok{ Fs, }\AttributeTok{xlab=}\StringTok{"Eye Closed"}\NormalTok{)}
\end{Highlighting}
\end{Shaded}

\includegraphics{lab4_sensor_XGboost_files/figure-latex/fft_closed-1.pdf}
\textbf{8} Do you see any differences between the power spectral
densities for the two eye states? If so, describe them.

Eyes Closed: Higher power in the lower frequency range, indicating more
prominent alpha wave activity.

Eyes Open: More uniform power distribution with generally lower power
levels, indicating reduced alpha wave activity.

\paragraph{Independent Component
Analysis}\label{independent-component-analysis}

We may also wish to explore whether there are multiple sources of
neuronal activity being picked up by the sensors.\\
This can be achieved using a process known as independent component
analysis (ICA) which decorrelates the channels and identifies the
primary sources of signal within the decorrelated matrix.

\begin{Shaded}
\begin{Highlighting}[]
\NormalTok{ica }\OtherTok{\textless{}{-}}\NormalTok{ eegkit}\SpecialCharTok{::}\FunctionTok{eegica}\NormalTok{(eeg\_train }\SpecialCharTok{\%\textgreater{}\%}\NormalTok{ dplyr}\SpecialCharTok{::}\FunctionTok{select}\NormalTok{(}\SpecialCharTok{{-}}\NormalTok{eyeDetection, }\SpecialCharTok{{-}}\NormalTok{ds), }\AttributeTok{nc=}\DecValTok{3}\NormalTok{, }\AttributeTok{method=}\StringTok{\textquotesingle{}fast\textquotesingle{}}\NormalTok{, }\AttributeTok{type=}\StringTok{\textquotesingle{}time\textquotesingle{}}\NormalTok{)}
\NormalTok{mix }\OtherTok{\textless{}{-}}\NormalTok{ dplyr}\SpecialCharTok{::}\FunctionTok{as\_tibble}\NormalTok{(ica}\SpecialCharTok{$}\NormalTok{M)}
\NormalTok{mix}\SpecialCharTok{$}\NormalTok{eyeDetection }\OtherTok{\textless{}{-}}\NormalTok{ eeg\_train}\SpecialCharTok{$}\NormalTok{eyeDetection}
\NormalTok{mix}\SpecialCharTok{$}\NormalTok{ds }\OtherTok{\textless{}{-}}\NormalTok{ eeg\_train}\SpecialCharTok{$}\NormalTok{ds}

\NormalTok{mix\_melt }\OtherTok{\textless{}{-}}\NormalTok{ reshape2}\SpecialCharTok{::}\FunctionTok{melt}\NormalTok{(mix, }\AttributeTok{id.vars=}\FunctionTok{c}\NormalTok{(}\StringTok{"eyeDetection"}\NormalTok{, }\StringTok{"ds"}\NormalTok{), }\AttributeTok{variable.name =} \StringTok{"Independent Component"}\NormalTok{, }\AttributeTok{value.name =} \StringTok{"M"}\NormalTok{)}


\NormalTok{ggplot2}\SpecialCharTok{::}\FunctionTok{ggplot}\NormalTok{(mix\_melt, ggplot2}\SpecialCharTok{::}\FunctionTok{aes}\NormalTok{(}\AttributeTok{x=}\NormalTok{ds, }\AttributeTok{y=}\NormalTok{M, }\AttributeTok{color=}\StringTok{\textasciigrave{}}\AttributeTok{Independent Component}\StringTok{\textasciigrave{}}\NormalTok{)) }\SpecialCharTok{+} 
\NormalTok{  ggplot2}\SpecialCharTok{::}\FunctionTok{geom\_line}\NormalTok{() }\SpecialCharTok{+} 
\NormalTok{  ggplot2}\SpecialCharTok{::}\FunctionTok{geom\_vline}\NormalTok{(ggplot2}\SpecialCharTok{::}\FunctionTok{aes}\NormalTok{(}\AttributeTok{xintercept=}\NormalTok{ds), }\AttributeTok{data=}\NormalTok{dplyr}\SpecialCharTok{::}\FunctionTok{filter}\NormalTok{(mix\_melt, eyeDetection}\SpecialCharTok{==}\DecValTok{1}\NormalTok{), }\AttributeTok{alpha=}\FloatTok{0.005}\NormalTok{) }\SpecialCharTok{+}
\NormalTok{  ggplot2}\SpecialCharTok{::}\FunctionTok{scale\_y\_log10}\NormalTok{()}
\end{Highlighting}
\end{Shaded}

\includegraphics{lab4_sensor_XGboost_files/figure-latex/ica-1.pdf}

\textbf{9} Does this suggest eye opening relates to an independent
component of activity across the electrodes?

Yes, the ICA analysis suggests that eye opening relates to an
independent component of activity across the electrodes. The plotted
independent components show distinct patterns when eyes are open,
indicated by vertical lines and changes in the activity levels, which
signifies that specific sources of neuronal activity are correlated with
the eye status.

\subsubsection{Eye Opening Prediction}\label{eye-opening-prediction}

Now that we've explored the data let's use a simple model to see how
well we can predict eye status from the EEGs:

\begin{Shaded}
\begin{Highlighting}[]
\CommentTok{\# Convert the training and validation datasets to matrices}
\NormalTok{eeg\_train\_matrix }\OtherTok{\textless{}{-}} \FunctionTok{as.matrix}\NormalTok{(dplyr}\SpecialCharTok{::}\FunctionTok{select}\NormalTok{(eeg\_train, }\SpecialCharTok{{-}}\NormalTok{eyeDetection, }\SpecialCharTok{{-}}\NormalTok{ds))}
\NormalTok{eeg\_train\_labels }\OtherTok{\textless{}{-}} \FunctionTok{as.numeric}\NormalTok{(eeg\_train}\SpecialCharTok{$}\NormalTok{eyeDetection) }\SpecialCharTok{{-}}\DecValTok{1}

\NormalTok{eeg\_validate\_matrix }\OtherTok{\textless{}{-}} \FunctionTok{as.matrix}\NormalTok{(dplyr}\SpecialCharTok{::}\FunctionTok{select}\NormalTok{(eeg\_validate, }\SpecialCharTok{{-}}\NormalTok{eyeDetection, }\SpecialCharTok{{-}}\NormalTok{ds))}
\NormalTok{eeg\_validate\_labels }\OtherTok{\textless{}{-}} \FunctionTok{as.numeric}\NormalTok{(eeg\_validate}\SpecialCharTok{$}\NormalTok{eyeDetection) }\SpecialCharTok{{-}}\DecValTok{1}

\CommentTok{\# Build the xgboost model}
\NormalTok{model }\OtherTok{\textless{}{-}} \FunctionTok{xgboost}\NormalTok{(}\AttributeTok{data =}\NormalTok{ eeg\_train\_matrix, }
                 \AttributeTok{label =}\NormalTok{ eeg\_train\_labels,}
                 \AttributeTok{nrounds =} \DecValTok{100}\NormalTok{,}
                 \AttributeTok{max\_depth =} \DecValTok{4}\NormalTok{,}
                 \AttributeTok{eta =} \FloatTok{0.1}\NormalTok{,}
                 \AttributeTok{objective =} \StringTok{"binary:logistic"}\NormalTok{)}
\end{Highlighting}
\end{Shaded}

\begin{verbatim}
## [1]  train-logloss:0.672173 
## [2]  train-logloss:0.653965 
## [3]  train-logloss:0.638226 
## [4]  train-logloss:0.622881 
## [5]  train-logloss:0.610129 
## [6]  train-logloss:0.599369 
## [7]  train-logloss:0.588913 
## [8]  train-logloss:0.577916 
## [9]  train-logloss:0.568707 
## [10] train-logloss:0.560877 
## [11] train-logloss:0.553595 
## [12] train-logloss:0.546697 
## [13] train-logloss:0.540356 
## [14] train-logloss:0.535189 
## [15] train-logloss:0.528949 
## [16] train-logloss:0.523818 
## [17] train-logloss:0.517499 
## [18] train-logloss:0.513480 
## [19] train-logloss:0.509431 
## [20] train-logloss:0.504572 
## [21] train-logloss:0.501298 
## [22] train-logloss:0.499068 
## [23] train-logloss:0.493927 
## [24] train-logloss:0.488979 
## [25] train-logloss:0.485995 
## [26] train-logloss:0.484120 
## [27] train-logloss:0.480735 
## [28] train-logloss:0.476749 
## [29] train-logloss:0.475324 
## [30] train-logloss:0.471852 
## [31] train-logloss:0.469037 
## [32] train-logloss:0.466092 
## [33] train-logloss:0.464552 
## [34] train-logloss:0.462219 
## [35] train-logloss:0.457720 
## [36] train-logloss:0.455526 
## [37] train-logloss:0.452435 
## [38] train-logloss:0.448676 
## [39] train-logloss:0.447381 
## [40] train-logloss:0.444740 
## [41] train-logloss:0.442885 
## [42] train-logloss:0.441704 
## [43] train-logloss:0.437273 
## [44] train-logloss:0.435778 
## [45] train-logloss:0.432542 
## [46] train-logloss:0.431302 
## [47] train-logloss:0.430229 
## [48] train-logloss:0.426916 
## [49] train-logloss:0.423800 
## [50] train-logloss:0.421908 
## [51] train-logloss:0.419130 
## [52] train-logloss:0.417934 
## [53] train-logloss:0.415003 
## [54] train-logloss:0.414027 
## [55] train-logloss:0.412499 
## [56] train-logloss:0.409956 
## [57] train-logloss:0.408821 
## [58] train-logloss:0.407170 
## [59] train-logloss:0.404487 
## [60] train-logloss:0.402856 
## [61] train-logloss:0.402061 
## [62] train-logloss:0.401169 
## [63] train-logloss:0.400292 
## [64] train-logloss:0.399799 
## [65] train-logloss:0.397281 
## [66] train-logloss:0.395909 
## [67] train-logloss:0.393773 
## [68] train-logloss:0.390365 
## [69] train-logloss:0.389440 
## [70] train-logloss:0.386978 
## [71] train-logloss:0.386324 
## [72] train-logloss:0.385750 
## [73] train-logloss:0.385101 
## [74] train-logloss:0.382760 
## [75] train-logloss:0.381081 
## [76] train-logloss:0.378908 
## [77] train-logloss:0.378428 
## [78] train-logloss:0.376087 
## [79] train-logloss:0.374926 
## [80] train-logloss:0.374230 
## [81] train-logloss:0.373538 
## [82] train-logloss:0.372971 
## [83] train-logloss:0.371651 
## [84] train-logloss:0.371134 
## [85] train-logloss:0.370717 
## [86] train-logloss:0.369246 
## [87] train-logloss:0.368063 
## [88] train-logloss:0.366157 
## [89] train-logloss:0.362902 
## [90] train-logloss:0.362461 
## [91] train-logloss:0.361028 
## [92] train-logloss:0.359356 
## [93] train-logloss:0.358512 
## [94] train-logloss:0.356873 
## [95] train-logloss:0.356331 
## [96] train-logloss:0.355519 
## [97] train-logloss:0.354799 
## [98] train-logloss:0.353089 
## [99] train-logloss:0.351400 
## [100]    train-logloss:0.350408
\end{verbatim}

\begin{Shaded}
\begin{Highlighting}[]
\FunctionTok{print}\NormalTok{(model)}
\end{Highlighting}
\end{Shaded}

\begin{verbatim}
## ##### xgb.Booster
## raw: 154.4 Kb 
## call:
##   xgb.train(params = params, data = dtrain, nrounds = nrounds, 
##     watchlist = watchlist, verbose = verbose, print_every_n = print_every_n, 
##     early_stopping_rounds = early_stopping_rounds, maximize = maximize, 
##     save_period = save_period, save_name = save_name, xgb_model = xgb_model, 
##     callbacks = callbacks, max_depth = 4, eta = 0.1, objective = "binary:logistic")
## params (as set within xgb.train):
##   max_depth = "4", eta = "0.1", objective = "binary:logistic", validate_parameters = "TRUE"
## xgb.attributes:
##   niter
## callbacks:
##   cb.print.evaluation(period = print_every_n)
##   cb.evaluation.log()
## # of features: 14 
## niter: 100
## nfeatures : 14 
## evaluation_log:
##      iter train_logloss
##     <num>         <num>
##         1     0.6721733
##         2     0.6539652
## ---                    
##        99     0.3514004
##       100     0.3504083
\end{verbatim}

\textbf{10} Using the \texttt{caret} library (or any other library/model
type you want such as a naive Bayes) fit another model to predict eye
opening.

\begin{Shaded}
\begin{Highlighting}[]
\FunctionTok{library}\NormalTok{(caret)}
\FunctionTok{library}\NormalTok{(randomForest)}
\end{Highlighting}
\end{Shaded}

\begin{verbatim}
## Warning: package 'randomForest' was built under R version 4.3.1
\end{verbatim}

\begin{verbatim}
## randomForest 4.7-1.1
\end{verbatim}

\begin{verbatim}
## Type rfNews() to see new features/changes/bug fixes.
\end{verbatim}

\begin{verbatim}
## 
## Attaching package: 'randomForest'
\end{verbatim}

\begin{verbatim}
## The following object is masked from 'package:dplyr':
## 
##     combine
\end{verbatim}

\begin{verbatim}
## The following object is masked from 'package:ggplot2':
## 
##     margin
\end{verbatim}

\begin{Shaded}
\begin{Highlighting}[]
\CommentTok{\# Prepare training and validation datasets}
\NormalTok{eeg\_train\_data }\OtherTok{\textless{}{-}}\NormalTok{ eeg\_train }\SpecialCharTok{\%\textgreater{}\%}\NormalTok{ dplyr}\SpecialCharTok{::}\FunctionTok{select}\NormalTok{(}\SpecialCharTok{{-}}\NormalTok{ds)}
\NormalTok{eeg\_validate\_data }\OtherTok{\textless{}{-}}\NormalTok{ eeg\_validate }\SpecialCharTok{\%\textgreater{}\%}\NormalTok{ dplyr}\SpecialCharTok{::}\FunctionTok{select}\NormalTok{(}\SpecialCharTok{{-}}\NormalTok{ds)}

\CommentTok{\# Convert eyeDetection to a factor for classification}
\NormalTok{eeg\_train\_data}\SpecialCharTok{$}\NormalTok{eyeDetection }\OtherTok{\textless{}{-}} \FunctionTok{as.factor}\NormalTok{(eeg\_train\_data}\SpecialCharTok{$}\NormalTok{eyeDetection)}
\NormalTok{eeg\_validate\_data}\SpecialCharTok{$}\NormalTok{eyeDetection }\OtherTok{\textless{}{-}} \FunctionTok{as.factor}\NormalTok{(eeg\_validate\_data}\SpecialCharTok{$}\NormalTok{eyeDetection)}

\CommentTok{\# Set up the train control for cross{-}validation}
\NormalTok{train\_control }\OtherTok{\textless{}{-}} \FunctionTok{trainControl}\NormalTok{(}\AttributeTok{method =} \StringTok{"cv"}\NormalTok{, }\AttributeTok{number =} \DecValTok{5}\NormalTok{)}

\CommentTok{\# Train the random forest model}
\FunctionTok{set.seed}\NormalTok{(}\DecValTok{123}\NormalTok{)}
\NormalTok{rf\_model }\OtherTok{\textless{}{-}} \FunctionTok{train}\NormalTok{(eyeDetection }\SpecialCharTok{\textasciitilde{}}\NormalTok{ ., }\AttributeTok{data =}\NormalTok{ eeg\_train\_data, }
                  \AttributeTok{method =} \StringTok{"rf"}\NormalTok{, }
                  \AttributeTok{trControl =}\NormalTok{ train\_control)}

\CommentTok{\# Print the model summary}
\FunctionTok{print}\NormalTok{(rf\_model)}
\end{Highlighting}
\end{Shaded}

\begin{verbatim}
## Random Forest 
## 
## 8988 samples
##   14 predictor
##    2 classes: '0', '1' 
## 
## No pre-processing
## Resampling: Cross-Validated (5 fold) 
## Summary of sample sizes: 7191, 7191, 7189, 7190, 7191 
## Resampling results across tuning parameters:
## 
##   mtry  Accuracy   Kappa    
##    2    0.9162239  0.8300647
##    8    0.9191161  0.8361384
##   14    0.9156671  0.8291978
## 
## Accuracy was used to select the optimal model using the largest value.
## The final value used for the model was mtry = 8.
\end{verbatim}

\begin{Shaded}
\begin{Highlighting}[]
\CommentTok{\# Predict on the validation data}
\NormalTok{rf\_predictions }\OtherTok{\textless{}{-}} \FunctionTok{predict}\NormalTok{(rf\_model, eeg\_validate\_data)}

\CommentTok{\# Calculate confusion matrix and accuracy}
\NormalTok{confusion\_matrix }\OtherTok{\textless{}{-}} \FunctionTok{confusionMatrix}\NormalTok{(rf\_predictions, eeg\_validate\_data}\SpecialCharTok{$}\NormalTok{eyeDetection)}
\FunctionTok{print}\NormalTok{(confusion\_matrix)}
\end{Highlighting}
\end{Shaded}

\begin{verbatim}
## Confusion Matrix and Statistics
## 
##           Reference
## Prediction    0    1
##          0 1536  138
##          1   99 1223
##                                           
##                Accuracy : 0.9209          
##                  95% CI : (0.9106, 0.9303)
##     No Information Rate : 0.5457          
##     P-Value [Acc > NIR] : < 2e-16         
##                                           
##                   Kappa : 0.8401          
##                                           
##  Mcnemar's Test P-Value : 0.01357         
##                                           
##             Sensitivity : 0.9394          
##             Specificity : 0.8986          
##          Pos Pred Value : 0.9176          
##          Neg Pred Value : 0.9251          
##              Prevalence : 0.5457          
##          Detection Rate : 0.5127          
##    Detection Prevalence : 0.5587          
##       Balanced Accuracy : 0.9190          
##                                           
##        'Positive' Class : 0               
## 
\end{verbatim}

\begin{Shaded}
\begin{Highlighting}[]
\CommentTok{\# Print overall accuracy}
\NormalTok{accuracy }\OtherTok{\textless{}{-}}\NormalTok{ confusion\_matrix}\SpecialCharTok{$}\NormalTok{overall[}\StringTok{\textquotesingle{}Accuracy\textquotesingle{}}\NormalTok{]}
\FunctionTok{print}\NormalTok{(accuracy)}
\end{Highlighting}
\end{Shaded}

\begin{verbatim}
##  Accuracy 
## 0.9208945
\end{verbatim}

\textbf{11} Using the best performing of the two models (on the
validation dataset) calculate and report the test performance (filling
in the code below):

\begin{Shaded}
\begin{Highlighting}[]
\FunctionTok{library}\NormalTok{(caret)}
\FunctionTok{library}\NormalTok{(xgboost)}
\FunctionTok{library}\NormalTok{(randomForest)}

\NormalTok{eeg\_train\_data }\OtherTok{\textless{}{-}}\NormalTok{ eeg\_train }\SpecialCharTok{\%\textgreater{}\%}\NormalTok{ dplyr}\SpecialCharTok{::}\FunctionTok{select}\NormalTok{(}\SpecialCharTok{{-}}\NormalTok{ds)}
\NormalTok{eeg\_validate\_data }\OtherTok{\textless{}{-}}\NormalTok{ eeg\_validate }\SpecialCharTok{\%\textgreater{}\%}\NormalTok{ dplyr}\SpecialCharTok{::}\FunctionTok{select}\NormalTok{(}\SpecialCharTok{{-}}\NormalTok{ds)}

\NormalTok{eeg\_train\_data}\SpecialCharTok{$}\NormalTok{eyeDetection }\OtherTok{\textless{}{-}} \FunctionTok{as.factor}\NormalTok{(eeg\_train\_data}\SpecialCharTok{$}\NormalTok{eyeDetection)}
\NormalTok{eeg\_validate\_data}\SpecialCharTok{$}\NormalTok{eyeDetection }\OtherTok{\textless{}{-}} \FunctionTok{as.factor}\NormalTok{(eeg\_validate\_data}\SpecialCharTok{$}\NormalTok{eyeDetection)}

\NormalTok{eeg\_train\_matrix }\OtherTok{\textless{}{-}} \FunctionTok{as.matrix}\NormalTok{(dplyr}\SpecialCharTok{::}\FunctionTok{select}\NormalTok{(eeg\_train, }\SpecialCharTok{{-}}\NormalTok{eyeDetection, }\SpecialCharTok{{-}}\NormalTok{ds))}
\NormalTok{eeg\_train\_labels }\OtherTok{\textless{}{-}} \FunctionTok{as.numeric}\NormalTok{(eeg\_train}\SpecialCharTok{$}\NormalTok{eyeDetection) }\SpecialCharTok{{-}} \DecValTok{1}

\NormalTok{eeg\_validate\_matrix }\OtherTok{\textless{}{-}} \FunctionTok{as.matrix}\NormalTok{(dplyr}\SpecialCharTok{::}\FunctionTok{select}\NormalTok{(eeg\_validate, }\SpecialCharTok{{-}}\NormalTok{eyeDetection, }\SpecialCharTok{{-}}\NormalTok{ds))}
\NormalTok{eeg\_validate\_labels }\OtherTok{\textless{}{-}} \FunctionTok{as.numeric}\NormalTok{(eeg\_validate}\SpecialCharTok{$}\NormalTok{eyeDetection) }\SpecialCharTok{{-}} \DecValTok{1}

\NormalTok{xgb\_model }\OtherTok{\textless{}{-}} \FunctionTok{xgboost}\NormalTok{(}\AttributeTok{data =}\NormalTok{ eeg\_train\_matrix, }
                     \AttributeTok{label =}\NormalTok{ eeg\_train\_labels,}
                     \AttributeTok{nrounds =} \DecValTok{100}\NormalTok{,}
                     \AttributeTok{max\_depth =} \DecValTok{4}\NormalTok{,}
                     \AttributeTok{eta =} \FloatTok{0.1}\NormalTok{,}
                     \AttributeTok{objective =} \StringTok{"binary:logistic"}\NormalTok{)}
\end{Highlighting}
\end{Shaded}

\begin{verbatim}
## [1]  train-logloss:0.672173 
## [2]  train-logloss:0.653965 
## [3]  train-logloss:0.638226 
## [4]  train-logloss:0.622881 
## [5]  train-logloss:0.610129 
## [6]  train-logloss:0.599369 
## [7]  train-logloss:0.588913 
## [8]  train-logloss:0.577916 
## [9]  train-logloss:0.568707 
## [10] train-logloss:0.560877 
## [11] train-logloss:0.553595 
## [12] train-logloss:0.546697 
## [13] train-logloss:0.540356 
## [14] train-logloss:0.535189 
## [15] train-logloss:0.528949 
## [16] train-logloss:0.523818 
## [17] train-logloss:0.517499 
## [18] train-logloss:0.513480 
## [19] train-logloss:0.509431 
## [20] train-logloss:0.504572 
## [21] train-logloss:0.501298 
## [22] train-logloss:0.499068 
## [23] train-logloss:0.493927 
## [24] train-logloss:0.488979 
## [25] train-logloss:0.485995 
## [26] train-logloss:0.484120 
## [27] train-logloss:0.480735 
## [28] train-logloss:0.476749 
## [29] train-logloss:0.475324 
## [30] train-logloss:0.471852 
## [31] train-logloss:0.469037 
## [32] train-logloss:0.466092 
## [33] train-logloss:0.464552 
## [34] train-logloss:0.462219 
## [35] train-logloss:0.457720 
## [36] train-logloss:0.455526 
## [37] train-logloss:0.452435 
## [38] train-logloss:0.448676 
## [39] train-logloss:0.447381 
## [40] train-logloss:0.444740 
## [41] train-logloss:0.442885 
## [42] train-logloss:0.441704 
## [43] train-logloss:0.437273 
## [44] train-logloss:0.435778 
## [45] train-logloss:0.432542 
## [46] train-logloss:0.431302 
## [47] train-logloss:0.430229 
## [48] train-logloss:0.426916 
## [49] train-logloss:0.423800 
## [50] train-logloss:0.421908 
## [51] train-logloss:0.419130 
## [52] train-logloss:0.417934 
## [53] train-logloss:0.415003 
## [54] train-logloss:0.414027 
## [55] train-logloss:0.412499 
## [56] train-logloss:0.409956 
## [57] train-logloss:0.408821 
## [58] train-logloss:0.407170 
## [59] train-logloss:0.404487 
## [60] train-logloss:0.402856 
## [61] train-logloss:0.402061 
## [62] train-logloss:0.401169 
## [63] train-logloss:0.400292 
## [64] train-logloss:0.399799 
## [65] train-logloss:0.397281 
## [66] train-logloss:0.395909 
## [67] train-logloss:0.393773 
## [68] train-logloss:0.390365 
## [69] train-logloss:0.389440 
## [70] train-logloss:0.386978 
## [71] train-logloss:0.386324 
## [72] train-logloss:0.385750 
## [73] train-logloss:0.385101 
## [74] train-logloss:0.382760 
## [75] train-logloss:0.381081 
## [76] train-logloss:0.378908 
## [77] train-logloss:0.378428 
## [78] train-logloss:0.376087 
## [79] train-logloss:0.374926 
## [80] train-logloss:0.374230 
## [81] train-logloss:0.373538 
## [82] train-logloss:0.372971 
## [83] train-logloss:0.371651 
## [84] train-logloss:0.371134 
## [85] train-logloss:0.370717 
## [86] train-logloss:0.369246 
## [87] train-logloss:0.368063 
## [88] train-logloss:0.366157 
## [89] train-logloss:0.362902 
## [90] train-logloss:0.362461 
## [91] train-logloss:0.361028 
## [92] train-logloss:0.359356 
## [93] train-logloss:0.358512 
## [94] train-logloss:0.356873 
## [95] train-logloss:0.356331 
## [96] train-logloss:0.355519 
## [97] train-logloss:0.354799 
## [98] train-logloss:0.353089 
## [99] train-logloss:0.351400 
## [100]    train-logloss:0.350408
\end{verbatim}

\begin{Shaded}
\begin{Highlighting}[]
\NormalTok{xgb\_predictions }\OtherTok{\textless{}{-}} \FunctionTok{predict}\NormalTok{(xgb\_model, eeg\_validate\_matrix)}
\NormalTok{xgb\_predictions }\OtherTok{\textless{}{-}} \FunctionTok{ifelse}\NormalTok{(xgb\_predictions }\SpecialCharTok{\textgreater{}} \FloatTok{0.5}\NormalTok{, }\DecValTok{1}\NormalTok{, }\DecValTok{0}\NormalTok{)}

\NormalTok{xgb\_confusion\_matrix }\OtherTok{\textless{}{-}} \FunctionTok{confusionMatrix}\NormalTok{(}\FunctionTok{as.factor}\NormalTok{(xgb\_predictions), }\FunctionTok{as.factor}\NormalTok{(eeg\_validate\_labels))}
\NormalTok{xgb\_accuracy }\OtherTok{\textless{}{-}}\NormalTok{ xgb\_confusion\_matrix}\SpecialCharTok{$}\NormalTok{overall[}\StringTok{\textquotesingle{}Accuracy\textquotesingle{}}\NormalTok{]}
\FunctionTok{print}\NormalTok{(xgb\_accuracy)}
\end{Highlighting}
\end{Shaded}

\begin{verbatim}
##  Accuracy 
## 0.8317757
\end{verbatim}

\begin{Shaded}
\begin{Highlighting}[]
\NormalTok{train\_control }\OtherTok{\textless{}{-}} \FunctionTok{trainControl}\NormalTok{(}\AttributeTok{method =} \StringTok{"cv"}\NormalTok{, }\AttributeTok{number =} \DecValTok{5}\NormalTok{)}

\FunctionTok{set.seed}\NormalTok{(}\DecValTok{123}\NormalTok{)}
\NormalTok{rf\_model }\OtherTok{\textless{}{-}} \FunctionTok{train}\NormalTok{(eyeDetection }\SpecialCharTok{\textasciitilde{}}\NormalTok{ ., }\AttributeTok{data =}\NormalTok{ eeg\_train\_data, }
                  \AttributeTok{method =} \StringTok{"rf"}\NormalTok{, }
                  \AttributeTok{trControl =}\NormalTok{ train\_control)}

\NormalTok{rf\_predictions }\OtherTok{\textless{}{-}} \FunctionTok{predict}\NormalTok{(rf\_model, eeg\_validate\_data)}

\NormalTok{rf\_confusion\_matrix }\OtherTok{\textless{}{-}} \FunctionTok{confusionMatrix}\NormalTok{(rf\_predictions, eeg\_validate\_data}\SpecialCharTok{$}\NormalTok{eyeDetection)}
\NormalTok{rf\_accuracy }\OtherTok{\textless{}{-}}\NormalTok{ rf\_confusion\_matrix}\SpecialCharTok{$}\NormalTok{overall[}\StringTok{\textquotesingle{}Accuracy\textquotesingle{}}\NormalTok{]}
\FunctionTok{print}\NormalTok{(rf\_accuracy)}
\end{Highlighting}
\end{Shaded}

\begin{verbatim}
##  Accuracy 
## 0.9208945
\end{verbatim}

\begin{Shaded}
\begin{Highlighting}[]
\CommentTok{\# Compare the accuracies and select the best model}
\ControlFlowTok{if}\NormalTok{ (xgb\_accuracy }\SpecialCharTok{\textgreater{}}\NormalTok{ rf\_accuracy) \{}
\NormalTok{  best\_model }\OtherTok{\textless{}{-}}\NormalTok{ xgb\_model}
\NormalTok{  best\_predictions }\OtherTok{\textless{}{-}}\NormalTok{ xgb\_predictions}
\NormalTok{  best\_confusion\_matrix }\OtherTok{\textless{}{-}}\NormalTok{ xgb\_confusion\_matrix}
\NormalTok{  best\_accuracy }\OtherTok{\textless{}{-}}\NormalTok{ xgb\_accuracy}
\NormalTok{\} }\ControlFlowTok{else}\NormalTok{ \{}
\NormalTok{  best\_model }\OtherTok{\textless{}{-}}\NormalTok{ rf\_model}
\NormalTok{  best\_predictions }\OtherTok{\textless{}{-}}\NormalTok{ rf\_predictions}
\NormalTok{  best\_confusion\_matrix }\OtherTok{\textless{}{-}}\NormalTok{ rf\_confusion\_matrix}
\NormalTok{  best\_accuracy }\OtherTok{\textless{}{-}}\NormalTok{ rf\_accuracy}
\NormalTok{\}}

\FunctionTok{print}\NormalTok{(best\_confusion\_matrix)}
\end{Highlighting}
\end{Shaded}

\begin{verbatim}
## Confusion Matrix and Statistics
## 
##           Reference
## Prediction    0    1
##          0 1536  138
##          1   99 1223
##                                           
##                Accuracy : 0.9209          
##                  95% CI : (0.9106, 0.9303)
##     No Information Rate : 0.5457          
##     P-Value [Acc > NIR] : < 2e-16         
##                                           
##                   Kappa : 0.8401          
##                                           
##  Mcnemar's Test P-Value : 0.01357         
##                                           
##             Sensitivity : 0.9394          
##             Specificity : 0.8986          
##          Pos Pred Value : 0.9176          
##          Neg Pred Value : 0.9251          
##              Prevalence : 0.5457          
##          Detection Rate : 0.5127          
##    Detection Prevalence : 0.5587          
##       Balanced Accuracy : 0.9190          
##                                           
##        'Positive' Class : 0               
## 
\end{verbatim}

\begin{Shaded}
\begin{Highlighting}[]
\FunctionTok{print}\NormalTok{(}\FunctionTok{paste}\NormalTok{(}\StringTok{"Best model accuracy: "}\NormalTok{, best\_accuracy))}
\end{Highlighting}
\end{Shaded}

\begin{verbatim}
## [1] "Best model accuracy:  0.920894526034713"
\end{verbatim}

\textbf{12} Describe 2 possible alternative modeling approaches for
prediction of eye opening from EEGs we discussed in the lecture but
haven't explored in this notebook.

Hidden Markov Models (HMMs):

These models describe the data as resulting from a series of hidden
states, where the observed EEG values are derived from these hidden
states. The transitions between states follow the Markov property,
meaning only the previous state(s) matter. HMMs are well-suited for
classification and detection tasks in time-series data, making them a
potential method for predicting eye opening based on EEG signals

Gaussian Processes: This non-parametric Bayesian approach models the
data by defining a distribution over all possible functions consistent
with the observed data. Gaussian Processes are characterized by a
covariance kernel, which can capture time, frequency, and state-space
models. They provide a flexible framework for predicting complex
patterns in EEG data, such as those associated with eye
opening\hspace{0pt}.

\textbf{13} What are 2 R libraries you could use to implement these
approaches? (note: you don't actually have to implement them though!)
HMMs - depmixS4: This package allows for the estimation and modeling of
hidden Markov models. It supports mixtures of various types of
distributions, making it suitable for analyzing EEG data.

Gaussian Processes - kernlab: This library supports a variety of
kernel-based learning methods, including Gaussian Processes for
regression and classification.

\subsection{Optional}\label{optional}

\textbf{14} (Optional) As this is the last practical of the course - let
me know how you would change future offerings of this course. This will
not impact your marks!

\begin{itemize}
\item
  What worked and didn't work for you (e.g., in terms of the practicals,
  tutorials, and lectures)?
\item
  Was learning how to run the practicals on your own machines instead of
  a clean server that will disappear after the course worth the
  technical challenges?
\item
  What would you add or remove from the course?
\item
  What was the main thing you will take away from this course?
\end{itemize}

\end{document}
